\chapter{muParserX library}
Library looked great but it had several disadvantages. I compiled it and some features were missing. I sent email with some minor code tweaks to Ingo Berg and he allowed me to cooperate with him on the project which was great news for me.

\fdocabbrevdeclare{RPN}{RPN}{reverse Polish notation}
muParserX supports all major compilers and is even compilable with \cpp{} 11. Main prerequisites were met and i started to cooperate on the project. Some important attributions done will be mentioned later. muParserX library offers many fancy features:
\begin{itemize}
\item support for scalar value types, matrices, boolean values and strings,
\item ability to write custom operators and functions,
\item ability to define variables and constants,
\item many other features, including \fdocabbrevref{RPN} representation if math formula,
\item ability to define functions with varying count of parameters.
\end{itemize}

These five facts were very important for me as they allowed me to start thinking about brand new calculator application that suits my needs. However, muParserX wasn't finished and rock-solid project. Many things needed to be solved. Ingo Berg led (and still leads) the way of muParserX development but two pairs of eyes see more. Code fixing, new ideas or debugging are important tasks too. Help is always needed.

Some functions (important for me) were missing in the library, so i simply added them. For example factorial function or percentage operator \% were added	and many other functions were rewritten or tweaked.

I participated in new design for internationalization which is massively significant for every developer. Typical user expects that software \enquote{talks} in his native language.

I changed behavior of definition of constants, variables, functions and operators along with other minor things.


